\section{Background}
\subsection{Word Embeddings}
A word embedding is a function mapping from words to their continuous vector representations.
The purpose of this method is to encode relevant syntactic and semantic features of words as relationships among their representation vectors.
Usually, word embeddings are trained on large, unlabeled corpus~\cite{glove,word2vec} and then fine-tuned with respect to specific NLP tasks~\cite{treeLSTM,KimCNN}.   
Pre-learned word presentations have been applied widely and become the ``secret sauce'' for the success of recent NLP systems~\cite{Luong_betterword}.
\paragraph{Matrix presentation of sequences of words}
Given any sentence \(s\) of length \(n\), we denote \(x_i \in \mathbb{R}^d\) as a vector presentation of word-\(i\)th in the sentence.
The first word in a sentence is word-\(0\)th.
In case the sentence is padded with dummy words on its left, these padded dummy words are indexed by negative integers.
Any sequence of words in the sentence which start at word-\(i\)th and end at word-\(j\)th can be presented as the following matrix:
\begin{align}
X_{i:j} &= x_i \oplus x_{i+1} \oplus ... \oplus x_{j} &\label{concat}
\end{align}
In Eq.\eqref{concat}, \(\oplus\) is concatenation operator which result in the matrix \(X_{i:j} \in \mathbb{R}^{d \times (j-i+1)}\).
\subsection{Convolution and Pooling}
Convolution Neural Networks (CNNs) have been proven to be effective models for the task of sentence-level sentiment analysis~\cite{KimCNN, DCNN,2-layer-cnn}.
Convolution Neural Networks usually constructed by staking up multiple convolution, pooling and fully connected layers.
\subsubsection{Convolution layer}
Given that \(F\) is the set of all filters of the convolution layer, for any filter \(v \in F\) which has window size \(l\) and set of parameters \(\theta^{(v)} = \{ W^{(v)}, b^{(v)} | W^{(v)} \in \mathbb{R}^{d \times l}, b^{(v)} \in \mathbb{R}\}\), filter \({v}\) is applied on any sequence of word-\(i\)th to word-\((i+l-1)\)th through the following equation:
\begin{align}
c^{(v)}_j &= f(W^{(v)} \otimes X_{i:i+l-1} + b^{(v)}) &\label{filter}
\end{align}
In Eq.\eqref{filter}, operator \(\otimes\) is the Hadamard product~\cite{element-prod}.  
\(b \in \mathbb{R}\) as bias term and \(f\) is an activation function.
For indexing, \(j = i + x\) with \(x \in \mathbb{N}\) and \(0 \leq x < l\).
If half-padding policy is employed then \(j = i + \floor{\frac{l}{2}}\).

By slicing the filter \(v\) through the sentence (i.e. applying the filter \(v\) on different sequences of length \(l\) along the sentence) we can get vector \(c^{(v)} = [c^{(v)}_0, c^{(v)}_1~\cdots]\) which is a feature map of the sentence~\(s\).
The length of the feature map \(c^{(v)}\) depends on the input sentence's length.
\subsubsection{Pooling layer}

Although being a simple Convolution Neural Network architect on top of pre-trained word embedding channels, CNN-multichannel~\cite{KimCNN} was able to archive state-of-the-art performance\footnote{2014} on Stanford Sentiment Treebank with binary setting.
Despite its success, for dealing with the problem composing fixed size presentation vectors given variable-length input sentences, CNN-multichannel utilized max-over-time pooling layer~\cite{nlp-scratch}.
\subsection{Recurrent Neural Network}
By definition, Recurrent Neural Networks are any Neural Networks which have at least a recurrent connection~\cite{rnn-def}.
\subsubsection{Vanilla Recurrent unit}\label{sec:vanilla-rnn}
Denoting input sequence as \(I = \{i_0,\ldots,i_n\}, \forall t, i_t \in \mathbb{R}^n\), Vanilla Recurrent unit can be expressed as the following recursive formula~\cite{treeLSTM}:
\begin{align}
h_t &= tanh(Wi_t + Uh_{t-1} + b)&\label{eq:rnn}
\end{align}
In Eq.\ref{eq:rnn}, \(h_t\) is called state of the system~\cite{deeplearning-book}, but we can also treat \(h_t\) as an output of the unit.

In theory, Vanilla Recurrent Neural Network is Turing-Complete~\cite{rnn-turing-complete} but  hard to train (especially on long input sequence) due to the problems of exploding and vanishing gradient~\cite{Bengio1994}.
In layman's terms, the longer the dependencies between an output and an input the (exponentially ) harder it is for the training process to capture those dependencies.
To mitigate the problem of vanishing gradient, Long Short Term Memory unit (LSTM)~\cite{originLSTM} unit was invented.
\subsubsection{Long Short Term Memory unit}\label{sec:lstm}
Given a input sequence \(I = (i_0,\ldots,i_n), \forall t, i_t \in \mathbb{R}^n\), Long Short Term Memory unit can be expressed as the following recursive formula~\cite{treeLSTM}:
\begin{align}
w_t &= \sigma(W^{(w)}i_t + U^{(w)}h_{t-1} + b^{(w)}) \label{eq:lstm-input-gate}&\\
f_t &= \sigma(W^{(f)}i_t + U^{(f)}h_{t-1} + b^{(f)}) \label{eq:lstm-forget-gate}&\\
o_t &= \sigma(W^{(o)}i_t + U^{(o)}h_{t-1} + b^{(o)}) \label{eq:lstm-output-gate}&\\
u_t &= tanh(W^{(u)}i_t + U^{(u)}h_{t-1} + b^{(u)}) \label{eq:lstm-update-gate}&\\
c_t &= r_t \otimes u_t + f_t \otimes c_{t-1} \label{eq:longterm-mem}&\\
h_t &= o_t \otimes tanh(c_t) \label{eq:temperal-mem}&
\end{align}
In the formula above, operator \(\otimes\) is the Hadamard product~\cite{element-prod}.
The role of \(h_t\) in LSTM unit is similar to its role in Vanilla Recurrent unit.
Traditionally, \(w_t\), \(f_t\) and \(o_t\) are called input/write gate, forget/deallocate gate and output/read gate respectively.
In addition, \(c_t\) is called memory cell.

Consistently, LSTM outperform Vanilla RNN in most tasks.
Intuitively, LSTM mitigates the problem of vanishing gradient by updating their memory through adding~\cite{evaluate-GRU}, although gradients still decrease through time, it is not exponentially~\cite{Graves-thesis}.
\subsection{Recursive Neural Network}
In most cases, to understand a sentence, we have to understand the phrases composing it.
Analogously, to understand a phrase, we have to understand the phrases and words composing it.
Recursive Neural Networks were mainly inspired by this idea~\cite{treeLSTM}.
Given a sentence and its parse tree, a Recursive Neural Network composes the vector presentation of the sentence by applying it composition function at each node of the parse tree in a bottom-up manner.
For demonstration, parse tree of the phrase ``is very interesting'' and its composing process using a Recursive Neural Network are illustrated in Fig.\ref{fig:example-parse} and Fig.\ref{fig:example-compose} respectively~\cite{tag-embedding-rnn}.


\begin{figure}% [H] meow note: H cause weird bug
	\centering
	% \includegraphics[scale=0.4]{figure/example-parse}
	%\resizebox{.4\textwidth}{!}{%
		\begin{tikzpicture}
		\node [sq] (v1) at (1,-1) {root};
		\node [sq] (v2) at (1,-2) {S};
		\node [sq] (v3) at (-1.5,-3) {NP};
		\node [sq] (v6) at (-1.5,-4) {PRP};
		\node [cir] (v7) at (-1.5,-5.5) {I};
		\node [sq] (v4) at (1,-3) {VP};
		\node [sq] (v8) at (0,-4) {VBP};
		\node [cir] (v10) at (0,-5.5) {feed};
		\node [cir] (v11) at (1.5,-5.5) {the};
		\node [cir] (v12) at (3,-5.5) {cat};
		\node [cir] (v13) at (4.5,-5.5) {.};
		\node [sq] (v9) at (2,-4) {NP};
		\node [sq] (v5) at (4.5,-3) {.};
		\draw [->] (v1) edge (v2);
		\draw [->] (v2) edge (v3);
		\draw [->] (v2) edge (v4);
		\draw [->] (v2) edge (v5);
		\draw [->] (v3) edge (v6);
		\draw [->] (v6) edge (v7);
		\draw [->] (v4) edge (v8);
		\draw [->] (v4) edge (v9);
		\draw [->] (v8) edge (v10);
		\draw [->] (v9) edge (v11);
		\draw [->] (v9) edge (v12);
		\draw [->] (v5) edge (v13);
		\end{tikzpicture}
	%}
	\caption[Constituency parse tree for the phrase ``is very interesting'']{Constituency parse tree for the phrase ``is very interesting''.}
	\label{fig:example-parse}
\end{figure}

\begin{figure} % [H] meow note: H cause weird bug
	\centering
	% \includegraphics[scale=0.3]{figure/example-compose}
	
\begin{tikzpicture}
\node [sqvec] (v1) at (3,0) {
	\node[cirsmall] {}; &\node[cirsmall] {}; &\node[cirsmall] {}; \\
};      
\node [sqvec] (v3) at (5,0) {
	\node[cirsmall] {}; &\node[cirsmall] {}; &\node[cirsmall] {}; \\
};      
\node [sqvec] (v4) at (4,2) {
	\node[cirsmall] {}; &\node[cirsmall] {}; &\node[cirsmall] {}; \\
};      
\node [sqvec] (v6) at (1,2) {
	\node[cirsmall] {}; &\node[cirsmall] {}; &\node[cirsmall] {}; \\
};      
\node [sqvec] (v8) at (-0.5,4) {
	\node[cirsmall] {}; &\node[cirsmall] {}; &\node[cirsmall] {}; \\
};      
\node [sqvec] (v7) at (2.5,4) {
	\node[cirsmall] {}; &\node[cirsmall] {}; &\node[cirsmall] {}; \\
};      
\node [sqvec] (v10) at (1,6) {
	\node[cirsmall] {}; &\node[cirsmall] {}; &\node[cirsmall] {}; \\
};      
\node [sq] (v2) at (4,1) {g};
\node [sq] (v5) at (2.5,3) {g};
\node [sq] (v9) at (1,5) {g};

\node at (-0.5,3.5) {I};
\node at (1,1.5) {feed};
\node at (3,-0.5) {the};
\node at (5,-0.5) {cat};
\node at (5,1.5) {the cat};
\node at (4,3.5) {feed the cat};
\node at (2.5,5.5) {I feed the cat};

\draw [arrow] (v1) edge (v2);
\draw [arrow] (v3) edge (v2);
\draw [arrow] (v2) edge (v4);
\draw [arrow] (v4) edge (v5);
\draw [arrow] (v6) edge (v5);
\draw [arrow] (v5) edge (v7);
\draw [arrow] (v8) edge (v9);
\draw [arrow] (v7) edge (v9);
\draw [arrow] (v9) edge (v10);
\end{tikzpicture}
	\caption[Applying Recursive Neural Network on the phrase ``is very interesting'']{Applying Recursive Neural Network on the Constituency parse tree of the phrase ``is very interesting''.
		The composition function of this network is denoted as \textbf{g}.}
	\label{fig:example-compose}
\end{figure}

\begin{figure} % [H] meow note: H cause weird bug
	\centering
	% \includegraphics[scale=0.3]{figure/example-compose}
	\begin{tikzpicture}
\node [sqvec,nodes={circ},      
every even row/.style = { nodes={fill=red!60}},
every odd row/.style = { nodes={fill=black!100}}] (c1) at (0.5,8.5) {
 \\
 \\ 
};  

\node [sqvec,nodes={circ},      
every even row/.style = { nodes={fill=red!60}},
every odd row/.style = { nodes={fill=black!100}}] (c2) at (1.5,8.5) {
 \\
 \\ 
};  

\node [sqvec,nodes={circ},      
every even row/.style = { nodes={fill=red!60}},
every odd row/.style = { nodes={fill=black!100}}] (c3) at (2.5,8.5) {
 \\
 \\ 
};  

\node [sqvec,nodes={circ},      
every even row/.style = { nodes={fill=red!60}},
every odd row/.style = { nodes={fill=black!100}}] (c4) at (3.5,8.5) {
 \\
 \\ 
};  

\node [sqvec,nodes={circ},      
every even row/.style = { nodes={fill=red!60}},
every odd row/.style = { nodes={fill=black!100}}] (c5) at (4.5,8.5) {
 \\
 \\ 
};  


\node [sqvec,column sep=-\pgflinewidth,nodes={sq1}] (v) at (2.5,4) {
&&&&\\
&&&&\\
&&&&\\
};   

\node [sqvec,column sep=-\pgflinewidth,nodes={sq1p}] (v1) at (-0.5,4) {
\\
\\
\\
};   
\node [sqvec,column sep=-\pgflinewidth,nodes={sq1p}] (v2) at (-1.5,4) {
\\
\\
\\
};   
\node [sqvec,column sep=-\pgflinewidth,nodes={sq1p}] (v3) at (5.5,4) {
\\
\\
\\
};   
\node [sqvec,column sep=-\pgflinewidth,nodes={sq1p}] (v4) at (6.5,4) {
\\
\\
\\
};   

\draw (v1-1-1.north west) -- (c1-2-1.west); % inner left
\draw (v-1-3.north west) -- (c1-2-1.east); % inner right
\draw (v2-1-1.north west) -- (c1-1-1.west); % outer left
\draw (v-1-4.north west) -- (c1-1-1.east); % outer right

\draw (v-1-1.north west) -- (c2-2-1.west); % inner left
\draw (v-1-4.north west) -- (c2-2-1.east); % inner right
\draw (v1-1-1.north west) -- (c2-1-1.west); % outer left
\draw (v-1-5.north west) -- (c2-1-1.east); % outer right

\draw (v-1-2.north west) -- (c3-2-1.west); % inner left
\draw (v-1-5.north west) -- (c3-2-1.east); % inner right
\draw (v-1-1.north west) -- (c3-1-1.west); % outer left
\draw (v-1-5.north east) -- (c3-1-1.east); % outer right

\draw (v-1-3.north west) -- (c4-2-1.west); % inner left
\draw (v-1-5.north east) -- (c4-2-1.east); % inner right
\draw (v-1-2.north west) -- (c4-1-1.west); % outer left
\draw (v3-1-1.north east) -- (c4-1-1.east); % outer right

\draw (v-1-4.north west) -- (c5-2-1.west); % inner left
\draw (v3-1-1.north east) -- (c5-2-1.east); % inner right
\draw (v-1-3.north west) -- (c5-1-1.west); % outer left
\draw (v4-1-1.north east) -- (c5-1-1.east); % outer right


\node [cir] (v8) at (2,12) {};
\node [cir] (v7) at (1,11) {};
\node [cir] (v6) at (3.5,11) {};
\node [cir] (v5) at (3,10) {};
\draw  (c3) edge (v5);
\draw  (c4) edge (v5);
\draw  (v5) edge (v6);
\draw  (c5) edge (v6);
\draw  (c1) edge (v7);
\draw  (v7) edge (v8);
\draw  (c2) edge (v7);
\draw  (v6) edge (v8);

\node at (-1.5,2) {pad};
\node at (-0.5,2) {pad};
\node at (0.5,2) {$x_1$};
\node at (1.5,2) {$x_2$};
\node at (2.5,2) {$x_3$};
\node at (3.5,2) {$x_4$};
\node at (4.5,2) {$x_5$};
\node at (5.5,2) {pad};
\node at (6.5,2) {pad};
\end{tikzpicture}
	\caption[mewo write sth here]{meow write sth here}
	\label{fig:cnntreelstm}
\end{figure}

The core idea behind the design of Tree-LSTMs is to generalize the LSTMs for tree-structured inputs.
Tree-LSTMs were able to archive state-of-the-art performance on two tasks: predicting the semantic relatedness of two sentences (SemEval 2014, Task 1~\cite{SemeEvalTask1}) and sentiment classification (Stanford Sentiment Treebank~\cite{socher2013recursive}).

Let \(d\) be the size of the input vectors, \(r
\) be the size of the memory cell and \(z\) be the number of sentiment classes. 

\paragraph{Leaf module}
Given any input vector \(x \in \mathbb{R}^d\), the calculation steps inside the leaf module can be expressed as follow:
\begin{align}
o &= \sigma{\left( W^{(o)} x + a^{\left(o\right)}\right)} & \\
c &= W^{(c)} x + a^{(c)} & \\
h &= o \odot \tanh{\left(c\right)} &
\end{align}

In this module, \(W^{(o)}, W^{(c)} \in \mathbb{R}^{r \times d}\) and \(a^{\left(o\right)}, a^{(c)} \in \mathbb{R}^r\).

\paragraph{Composer module}
Given the input vectors \({h_l}\), \({c_l}\) from the left child node and \({h_r}\), \({c_r}\) from the right child node, the calculation steps inside the composer module can be expressed as follow:
\begin{align}
i &= \sigma{ \left(U_l^{(i)} h_{l} + U_r^{(i)} h_{r} + b^{(i)} \right) } &\\
f_{l} &= \sigma{\left(U_{l}^{(l)} h_{l} + U_{r}^{(l)} h_{r} + b^{(f)}\right)} & \\
f_{r} &= \sigma{\left(U_{l}^{(r)} h_{l} + U_{r}^{(r)} h_{r} + b^{(f)}\right)} & \\
o &= \sigma{\left( U_l^{(o)} h_{l} + U_r^{(o)} h_{r} + b^{(o)}\right)} &\\
u &= \tanh{\left( U_l^{(u)} h_{l} + U_r^{(u)} h_{r} + b^{(u)}\right)} &\\
c &= i \odot u + f_{l} \odot c_{l} + f_{r} \odot c_{r} & \\
h &= o \odot \tanh{\left(c\right)} &
\end{align}

In this module, for any \(j \in \{i, l, r, o, u\}\) and \(x \in \{\l, r\}\), \(U_x^{(j)} \in \mathbb{R}^{r \times r}\) and \( b^{(j)} \in \mathbb{R}^r\).

\paragraph{Composing sentence}
Given any sentence \({s}\) of length \({n}\) and its parse tree, originally, Tree-LSTM composes the output vectors by first applying its leaf module on each word vectors then applying its composition function at each node of the parse tree in a bottom-up manner.
